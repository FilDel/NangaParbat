\documentclass[10pt,a4paper]{article}
\usepackage{amsmath,amssymb,bm,makeidx,subfigure}
\usepackage[italian,english]{babel}
\usepackage[center,small]{caption}[2007/01/07]
\usepackage{fancyhdr}
\usepackage{color}

\definecolor{blu}{rgb}{0,0,1}
\definecolor{verde}{rgb}{0,1,0}
\definecolor{rosso}{rgb}{1,0,0}
\definecolor{viola}{rgb}{1,0,1}
\definecolor{arancio}{rgb}{1,0.5,0}
\definecolor{celeste}{rgb}{0,1,1}
\definecolor{rosa}{rgb}{1,0.3,0.5}

\oddsidemargin = 12pt
\topmargin = 0pt
\textwidth = 440pt
\textheight = 650pt

\makeindex

\begin{document}

\section{Structure of the observables}

Let us start from Eq.~(2.6) of Ref.~\cite{Scimemi:2017etj}, that is
the fully differential cross section for lepton-pair production in the
region in which the TMD factorisation applies, $i.e.$ $q_T \ll
Q$. After some minor manipulations, it reads:
\begin{equation}\label{eq:crosssection}
\frac{d\sigma}{dQ dy dq_T} =
\frac{16\pi\alpha^2(Q)q_T\mathcal{P}(q_T,Q)}{3N_c Q^3} H(Q,\mu) \sum_q C_q(Q)
\int\frac{d^2\mathbf{b}}{4\pi} e^{i \mathbf{b}\cdot \mathbf{q}_T} x_1F_q(x_1,\mathbf{b};\mu,\zeta) x_2F_{\bar{q}}(x_2,\mathbf{b};\mu,\zeta)\,,
\end{equation}
where $Q$, $y$, and $q_T$ are the invariant mass, the rapidity, and
the transverse momentum of the lepton pair, respectively, while
$N_c=3$ is the number of colours, $\alpha$ is the electromagnetic
coupling, $H$ is the appropriate QCD form factor that can be
perturbatively computed, and $C_q$ are the effective electroweak
charges. In addition, the variables $x_1$ and $x_2$ are functions of
$Q$ and $y$ and are given by:
\begin{equation}\label{eq:Bjorkenx12}
  x_{1,2} = \frac{Q}{\sqrt{s}}e^{\pm y}\,,
\end{equation}
being $\sqrt{s}$ the centre-of-mass energy of the collision. The
kinematic factor $\mathcal{P}$ takes into account the reduction of the
integration leptonic phase space due to possible cuts on the leptons
and thus it depends on $q_T$, $y$, and $Q$ as well as on the numerical
values of the cut parameters. Finally, the scales $\mu$ and $\zeta$
are introduced through TMD factorisation to factorise collinear and
rapidity divergences. As usual, despite they are arbitrary scales,
they are typically chosen $\mu=\sqrt{\zeta}=Q$. Therefore, for all
practical purposes their presence is fictitious.

The computation-intensive part of eq.(\ref{eq:crosssection}) has the
form of the integral:
\begin{equation}\label{eq:integral}
I_{ij}(x_1,x_2,q_T;\mu,\zeta)=\int\frac{d^2\mathbf{b}}{4\pi} e^{i \mathbf{b}\cdot \mathbf{q}_T} x_1F_i(x_1,\mathbf{b};\mu,\zeta) x_2F_{j}(x_2,\mathbf{b};\mu,\zeta)\,.
\end{equation}
where $F_{i(j)}$ are combinations of evolved TMD PDFs. At this stage,
for convenience, $i$ and $j$ do not coincide with $q$ and $\bar{q}$
but they are linked through a simple linear transformation. The
integral over the bidimensional impact parameter \textbf{b} has to be
taken. However, $F_{i(j)}$ only depend on the absolute value of
\textbf{b}, therefore eq.~(\ref{eq:integral}) can be written as:
\begin{equation}\label{eq:integral2}
I_{ij}(x_1,x_2,q_T;\mu,\zeta)=\frac12\int_0^\infty db\,b J_0(bq_T)  x_1
F_i(x_1,b;\mu,\zeta) x_2 F_{j}(x_2,b;\mu,\zeta)\,.
\end{equation}
where $J_0$ is the zero-th order Bessel function of the first kind
whose integral representation is:
\begin{equation}
J_0(x) = \frac1{2\pi}\int_0^{2\pi} d\theta e^{ix\cos(\theta)}\,.
\end{equation}
The single evolved TMD PDF $F_i$ at the final scales $\mu$ and $\zeta$
is obtained by multiplying the same TMD PDF at the initial scales
$\mu_0$ and $\zeta_0$ by a single evolution factor
$R_q$(\footnote{Note that in eq.~(\ref{eq:crosssection}) the gluon TMD
  PDF $F_g$ is not involved. If also the gluon TMD PDF was involved,
  it would evolve by means of a different evolution factor $R_g$.}),
that is:
\begin{equation}
  xF_i(x,b;\mu,\zeta) = R_q(\mu_0,\zeta_0\rightarrow \mu,\zeta;b)
  xF_i(x,b;\mu_0,\zeta_0)\,.
\end{equation}
so that eq.~(\ref{eq:integral2}) becomes:
\begin{equation}\label{eq:integral3}
I_{ij}(x_1,x_2,q_T;\mu,\zeta)=\frac12\int_0^\infty db\,b J_0(bq_T)
\left[R_q(\mu_0,\zeta_0\rightarrow \mu,\zeta;b)\right]^2 x_1 F_i(x_1,b;\mu_0,\zeta_0) x_2F_{j}(x_2,b;\mu_0,\zeta_0)\,.
\end{equation}

The initial scale TMD PDFs at LO in the OPE region, that is for
$b\ll B$ where $B$ is an unknown non-perturbative parameter that
represents the intrinsic hadron scale (see eq.~(2.27) of
Ref.~\cite{Scimemi:2017etj}), can be written as:
\begin{equation}\label{eq:LOconv}
xF_i(x,b;\mu_0,\zeta_0) = \sum_{j=g,q(\bar{q})}x\int_x^1\frac{dy}{y}C_{ij}(y;\mu_0,\zeta_0)f_j\left(\frac{x}{y},\mu_0\right)\,,
\end{equation}
where $f_j$ are the collinear PDFs (including the gluon) and $C_{ij}$
are the so-called matching functions that are perturbatively
computable and are currently known to NNLO, $i.e.$
$\mathcal{O}(\alpha_s^2)$. If we define:
\begin{equation}
\widetilde{f}_i\left(x,\mu_0\right) = xf_i\left(x,\mu_0\right)\,,
\end{equation}
eq.~(\ref{eq:LOconv}) can be written as:
\begin{equation}\label{eq:LOconvNPbl}
x F_i(x,b;\mu_0,\zeta_0) =
\sum_{j=g,q(\bar{q})}\int_x^1dy\,C_{ij}(y;\mu_0,\zeta_0)  \widetilde{f}_i\left(\frac{x}{y},\mu_0\right)\,.
\end{equation}
At this point, it is opportune to mention that the variables $\mu_0$
and $\zeta_0$ are usually taken to be functions of the impact
parameter $b$. Therefore, eq.~(\ref{eq:LOconvNPbl}) is a function of
two variables only that we rewrite way as:
\begin{equation}\label{eq:LOconvNPsimp}
  x F_i(x,b) =
  \sum_{j=g,q(\bar{q})}\int_x^1dy\,C_{ij}(y,b)  \widetilde{f}_i\left(\frac{x}{y},b\right)\,.
\end{equation}

This kind of convolutions can be computed using standard interpolation
techniques by which one approximates the function $\widetilde{f}_i$ as:
\begin{equation}
  \widetilde{f}_i(x,b) = \sum_\alpha w_\alpha(x) \widetilde{f}_i(x_\alpha,b)
\end{equation}
where $x_\alpha$ is the $\alpha$-th node of an interpolation grid and
$w_\alpha$ is the interpolating function associated to that
node. Assuming for now that $x$ coincides with the $\beta$-th node of
the grid and introducing another grid in the $b$ dimension whose nodes
are indexed by $\tau$, eq.~(\ref{eq:LOconvNPsimp}) can be written as:
\begin{equation}\label{eq:LOconvNPsimp1}
  \hat{F}_{i,\beta}^\tau = \sum_j\sum_\alpha\hat{C}_{ij,\beta\alpha}^\tau \hat{f}_{j,\alpha}^\tau\,.
\end{equation}
where we have used the following definitions:
\begin{equation}
\hat{F}_{i,\beta}^\tau\equiv x_\beta F_i(x_\beta,b_\tau)\,,\quad
\hat{C}_{ij,\beta\alpha}^\tau \equiv
\int_{x_\beta}^1dy\,C_{ij}(y,b_\tau)w_\alpha\left(\frac{x_\beta}{y}\right)\,,\quad
  \hat{f}_{j,\alpha}^\tau \equiv \widetilde{f}_i(x_\alpha,b_\tau)\,.
\end{equation}

Since we have to integrate over the impact parameter $b$ (see
eq.~(\ref{eq:integral2})), we need to be able to reconstruct the \
dependence of the function $F_i$ on $b$. This can be done using the
same interpolation technique. In particular, we write:
\begin{equation}
  x_\alpha F_i(x_{\alpha},b)
  x_\beta F_j(x_\beta,b) = 
  \sum_\tau \widetilde{w}_{\tau}(b) \hat{F}_{i,\alpha}^\tau \hat{F}_{j,\beta}^\tau
  = \sum_\tau \widetilde{w}_{\tau}(b) \sum_{kl}\sum_{\gamma\delta} \hat{C}_{ik,\alpha\gamma}^\tau
  \hat{C}_{jl,\beta\delta}^\tau \hat{f}_{k,\gamma}^\tau \hat{f}_{l,\delta}^\tau\,.
\end{equation}
Keeping in mind that $\mu_0$ and $\zeta_0$ are functions of the impact
parameter $b$ and that $\mu=\sqrt{\zeta} =Q$, eq.~(\ref{eq:integral3})
takes the form:
\begin{equation}\label{eq:lumiInter}
  I_{ij}(x_\alpha,x_\beta,q_T;Q)= \sum_\tau K_\tau(Q;q_T)
  \sum_{kl}\sum_{\gamma\delta} \hat{C}_{ik,\alpha\gamma}^\tau
  \hat{C}_{jl,\beta\delta}^\tau \hat{f}_{k,\gamma}^\tau \hat{f}_{l,\delta}^\tau\,,
\end{equation}
where we have defined:
\begin{equation}\label{eq:Kcoeff}
  K_\tau(Q;q_T)\equiv\frac12\int_0^\infty db\,b J_0(bq_T)
  \left[R_q(Q;b)\right]^2 \widetilde{w}_{\tau}(b)\,,
\end{equation}
being $R_q(Q;b)\equiv R_q(\mu_0,\zeta_0\rightarrow \mu,\zeta;b)$. It
should be noticed that $\widetilde{w}_{\tau}$ is a piecewise function
different from zero only over a finite interval in $b$, say
$[c_\tau,d_\tau]$. In practice, $\widetilde{w}_{\tau}$ extends over
$k+1$ intervals on the grid in $b$, being $k$ the interpolation
degree, around the node $b_\tau$ so that, typically
$c_\tau=b_{\tau-k}$ and $d_\tau=b_{\tau+1}$. Therefore the integral in
eq.~(\ref{eq:Kcoeff}) reduces to:
\begin{equation}\label{eq:Kcoeff1}
  K_\tau(Q;q_T)\equiv\frac12\int_{c_\tau}^{d_\tau} db\,b J_0(bq_T)
  \left[R_q(Q;b)\right]^2 \widetilde{w}_{\tau}(b)\,,
\end{equation}
and the sum over $\tau$ in eq.~(\ref{eq:lumiInter}), that is supposed
to run over an infinite number of nodes, has to be truncated.

As customary in QCD, the most convenient flavour basis, that is the
one that minimises the mixing between operators, is the so-called
``evolution'' basis (\textit{i.e.} $\Sigma$, $V$, $T_3$, $V_3$,
etc.). In fact, in this basis the operators matrix $C_{ij}$ is almost
diagonal with the only exception of crossing terms that couple the
gluon and the singlet $\Sigma$ distributions. This greatly simplifies
the sums over $k$ and $l$ in eq.~(\ref{eq:lumiInter}). On the other
hand, given that the TMDs that appear in eq.~(\ref{eq:crosssection})
are in the so-called ``physical'' basis (\textit{i.e.} $d$, $\bar{d}$,
$u$, $\bar{u}$, etc.), we need to rotate the quantity in
eq.~(\ref{eq:lumiInter}) from the evolution basis, over which the
indices $i$ and $j$ run, to the physical basis. This is done by means
of an appropriate constant matrix $T$, so that:
\begin{equation}\label{eq:lumiInterRot}
I_{q\bar{q}}(x_\alpha,x_\beta,q_T;Q)= \sum_\tau \sum_{kl} \sum_{\gamma\delta}\sum_{ij}K_\tau(Q;q_T) T_{qi}T_{\bar{q}j}\hat{C}_{ik,\alpha\gamma}^\tau
  \hat{C}_{jl,\beta\delta}^\tau \hat{f}_{k,\gamma}^\tau \hat{f}_{l,\delta}^\tau\,.
\end{equation}

In order account for higher orders in the OPE and non-perturbative
effects where the OPE is not valid, one usually introduces a
phenomenological non-perturbative function $f_{\rm NP}$ that modifies
the convolution in eq.~(\ref{eq:LOconv}) The way how $f_{\rm NP}(x,b)$
is introduced in not unique. Here we choose to follow the most
traditional approach in which TMDs get corrected by a multiplicative
function, that is to say:
\begin{equation}\label{eq:LOconvNP1}
  xF_i(x,b) \rightarrow f_{\rm NP}(x,b) xF_i(x,b)\,.
\end{equation}
This can be easily introduced in eq.~(\ref{eq:x1x2inter}) by defining:
\begin{equation}
f_{{\rm NP},\alpha}^\tau \equiv f_{\rm NP}(x_\alpha,b_\tau)\,,
\end{equation}
so that:
\begin{equation}\label{eq:lumiInterRotNP}
I_{q\bar{q}}(x_\alpha,x_\beta,q_T;Q)= \sum_\tau \sum_{kl} \sum_{\gamma\delta}\sum_{ij}K_\tau(Q;q_T) T_{qi}T_{\bar{q}j}\hat{C}_{ik,\alpha\gamma}^\tau
  \hat{C}_{jl,\beta\delta}^\tau \hat{f}_{k,\gamma}^\tau \hat{f}_{l,\delta}^\tau f_{{\rm NP},\alpha}^\tau f_{{\rm NP},\beta}^\tau\,.
\end{equation}
The computation of $I_{q\bar{q}}$ for a generic $x_1$ and $x_2$ is
achieved by interpolation as:
\begin{equation}\label{eq:x1x2inter}
\begin{array}{c}
  \displaystyle I_{q\bar{q}}(x_1,x_2,q_T;Q) =
  \sum_{\alpha\beta}w_\alpha(x_1)w_\beta(x_2)I_{q\bar{q}}(x_\alpha,x_\beta,q_T;Q)
  =\\
\\
\displaystyle \sum_\tau\sum_{\alpha\beta} \sum_{kl} \sum_{\gamma\delta}\sum_{ij}K_\tau(Q;q_T)w_\alpha(x_1)w_\beta(x_2) T_{qi}T_{\bar{q}j}\hat{C}_{ik,\alpha\gamma}^\tau
  \hat{C}_{jl,\beta\delta}^\tau \hat{f}_{k,\gamma}^\tau \hat{f}_{l,\delta}^\tau f_{{\rm NP},\alpha}^\tau f_{{\rm NP},\beta}^\tau\,.
\end{array}
\end{equation}

Keeping in mind eq.~(\ref{eq:Bjorkenx12}), one realises that the
variables $x_1$ and $x_2$ are functions of $Q$ and $y$ and thus one
can simply write:
\begin{equation}\label{eq:finalres}
  I_{q\bar{q}}(Q, y, q_T) =
  \sum_\tau \sum_{\alpha\beta} W_{q\bar{q},\alpha\beta}^{\tau}(Q,y,q_T) f_{{\rm NP},\alpha}^\tau f_{{\rm NP},\beta}^\tau\,,
\end{equation}
where we have defined:
\begin{equation}
W_{q\bar{q},\alpha\beta}^{\tau}(Q,y,q_T)\equiv \sum_{kl} \sum_{\gamma\delta}\sum_{ij}K_\tau(Q;q_T) w_\alpha\left(\frac{Q}{\sqrt{s}}e^y\right)w_\beta\left(\frac{Q}{\sqrt{s}}e^{-y}\right) T_{qi}T_{\bar{q}j}\hat{C}_{ik,\alpha\gamma}^\tau
  \hat{C}_{jl,\beta\delta}^\tau \hat{f}_{k,\gamma}^\tau \hat{f}_{l,\delta}^\tau\,.
\end{equation}
Unsurprisingly, the $W$ factors can be factorised as:
\begin{equation}
  W_{q\bar{q},\alpha\beta}^{\tau}(Q,y,q_T)\equiv
  K_\tau(Q;q_T) w_\alpha\left(\frac{Q}{\sqrt{s}}e^y\right) \left(\sum_{i} T_{qi}\sum_{k} \sum_{\gamma}\hat{C}_{ik,\alpha\gamma}^\tau
    \hat{f}_{k,\gamma}^\tau\right)
w_\beta\left(\frac{Q}{\sqrt{s}}e^{-y}\right)  \left(\sum_{j} T_{\bar{q}j}\sum_{l} \sum_{\delta}
    \hat{C}_{jl,\beta\delta}^\tau \hat{f}_{l,\delta}^\tau\right)\,.
\end{equation}

This equation emphasises that $I_{q\bar{q}}$ is a function of three
independent kinematics variables $Q$, $y$, and $q_T$. This is relevant
when integrating the cross section over the experimental bins as we
will discuss in the next section. With eq.~(\ref{eq:finalres}) at
hand, eq.~(\ref{eq:crosssection}) can be written as:
\begin{equation}
\frac{d\sigma}{dQ dy dq_T} =\sum_\tau \sum_{\alpha\beta}\left[
  \frac{16\pi\alpha^2(Q)q_T\mathcal{P}(q_T,Q)}{3N_c Q^3} H(Q) 
  \sum_q C_q(Q) W_{q\bar{q},\alpha\beta}^{\tau}(Q,y,q_T) \right]f_{{\rm NP},\alpha}^\tau f_{{\rm NP},\beta}^\tau\,.
\end{equation}
Crucially, the quantity inside the squared brackets is fully
determined by the kinematics and the leading-twist component of the
process, while the non-perturbative part is fully factorised. Clearly,
this is extremely useful if one wants to fit the non-perturbative
component to data.

\section{Integrating over the final-state kinematic variables}

Despite eq.~(\ref{eq:lumiInterRot}) provides a powerful tool for a
fast computation of cross sections, it is often not sufficient to
allow for a direct comparison to experimental data. The reason is that
experimental measurements of differential distributions are usually
delivered integrated over finite regions of the final-state kinematic
phase space. In other words, experiments measure quantities like:
\begin{equation}\label{eq:Intcrosssection}
\widetilde{\sigma}=\int_{Q_{\rm min}}^{Q_{\rm max}}dQ \int_{y_{\rm min}}^{y_{\rm max}}dy \int_{q_{T,\rm min}}^{q_{T,\rm max}}dq_T\left[\frac{d\sigma}{dQ dy dq_T} \right]\,.
\end{equation}
As a consequence, in order to guarantee performance, we need to
include the integrations above in the precomputed factors.

\subsection{Integrating over $q_T$}

The integration over $q_T$ is relatively simple to implement because
the full dependence on $q_T$ in eq.~(\ref{eq:Kcoeff1}) is given by the
factors $q_T$, $\mathcal{P}$, and $K_\tau$. Therefore, integrating
over $q_T$ simply amounts of computing the integrals:
\begin{equation}\label{eq:KcoeffInt}
  \widetilde{K}_\tau(Q)\equiv \int_{q_{T,\rm min}}^{q_{T,\rm max}}dq_T\,q_T\mathcal{P}(q_T,Q) K_\tau(Q;q_T)
\end{equation}

\subsection{Integrating over $y$}

The dependence on $y$ of the cross section in
eq.~(\ref{eq:crosssection}) exclusively happens through the variables
$x_1$ and $x_2$ defined in eq.~(\ref{eq:Bjorkenx12}). Since this
dependence is reconstructed through interpolation in
eq.~(\ref{eq:x1x2inter}), what we need to do is computing the
following integrals:
\begin{equation}
u_{\alpha\beta}(Q)\equiv \int_{y_{\rm min}}^{y_{\rm max}}dy\,w_\alpha\left(\frac{Q}{\sqrt{s}}e^y\right)w_\beta\left(\frac{Q}{\sqrt{s}}e^{-y}\right)
\end{equation}
and replace $w_\alpha(x_1)w_\beta(x_2)$ in eq.~(\ref{eq:x1x2inter})
with $u_{\alpha\beta}(Q)$.

\subsection{Integrating over $Q$}

The integration over $Q$ has finally to be done by brute force due to
the fact that the dependence on $Q$ of the expression we are
considering is not localised and involves essentially all ingredients
(we remind that we are assuming $\mu=\sqrt{\zeta}=Q$). One alternative
solution is to use the so-called narrow-width approximation (NWA) in
which one assumes that the width of the $Z$ boson $\Gamma_Z$ is much
smaller that its mass $M_Z$. This way one can approximate the peaked
behaviour of the couplings $C_q(Q)$ around $Q=M_Z$ with a
$\delta$-function, \textit{i.e.} $C_q(Q)\sim \delta(Q^2-M_Z^2)$, so
that the integration over $Q$ can be done analytically essentially
setting $Q=M_Z$ everywhere in the expression. This approximation,
though, is usable only for data around the $Z$ peak and, of course, it
is only an approximation and thus might lead to substantial
inaccuracies. Therefore, it is useful to be able to carry out the
integration over $Q$ explicitly.

To this end, we start by writing explicitly the cross section
integrated over $q_T$ and $y$ making use of the definitions given in
the previous subsections:
\begin{equation}
  \frac{d\sigma}{dQ} =\sum_\tau \sum_{\alpha\beta}\left[\frac{16\pi}{3N_c}\sum_{kl} \sum_{\gamma\delta}\sum_{ij}\sum_q
    \left(\frac{\alpha^2(Q)}{Q^3} H(Q) u_{\alpha\beta}(Q)
    \widetilde{K}_\tau(Q) C_q(Q)\right)
    T_{qi}T_{\bar{q}j}
    \hat{C}_{ik,\alpha\gamma}^\tau \hat{C}_{jl,\beta\delta}^\tau 
  \hat{f}_{k,\gamma}^{\tau}
  \hat{f}_{l,\delta}^{\tau}\right]f_{{\rm NP},\alpha}^\tau f_{{\rm NP},\beta}^\tau\,,
\end{equation}
where we have purposely enclosed between round brakets the
$Q$-dependant factors. In fact, if we define:
\begin{equation}
S_{q,\alpha\beta}^{\tau}\equiv \frac{16\pi}{3N_c}\int_{Q_{\rm min}}^{Q_{\rm max}}dQ\,
    \frac{\alpha^2(Q)}{Q^3} H(Q) \widetilde{K}_\tau(Q) u_{\alpha\beta}(Q)
    C_q(Q)\,,
\end{equation}
we have that:
\begin{equation}\label{eq:FullInt}
  \widetilde{\sigma} =\sum_\tau \sum_{\alpha\beta}\left[ \sum_q
    S_{q,\alpha\beta}^{\tau} \left(\sum_{i}
    T_{qi}\sum_{k} \sum_{\gamma}
    \hat{C}_{ik,\alpha\gamma}^\tau 
  \hat{f}_{k,\gamma}^{\tau}\right)
\left(\sum_{j}
    T_{\bar{q}j}\sum_{l} \sum_{\delta}
    \hat{C}_{jl,\beta\delta}^\tau 
  \hat{f}_{l,\delta}^{\tau}\right)
\right]f_{{\rm NP},\alpha}^\tau f_{{\rm NP},\beta}^\tau\,.
\end{equation}
so that, defining:
\begin{equation}
  \overline{F}_{q(\bar{q}),\alpha}^\tau \equiv \sum_{i}
  T_{q(\bar{q})i}\sum_{k} \sum_{\gamma}
  \hat{C}_{ik,\alpha\gamma}^\tau 
  \hat{f}_{k,\gamma}^{\tau}\,,
\end{equation}
eq.~(\ref{eq:FullInt}) can be readily written as:
\begin{equation}
  \widetilde{\sigma} =\sum_\tau \sum_{\alpha\beta}\left[ \sum_q
    S_{q,\alpha\beta}^{\tau}\overline{F}_{q,\alpha}^\tau
    \overline{F}_{\bar{q},\beta}^\tau \right]f_{{\rm NP},\alpha}^\tau
  f_{{\rm NP},\beta}^\tau=\sum_\tau
  \sum_{\alpha\beta}M_{\alpha\beta}^\tau f_{{\rm NP},\alpha}^\tau f_{{\rm NP},\beta}^\tau\,,
\end{equation}
with:
\begin{equation}
M_{\alpha\beta}^\tau\equiv \sum_q S_{q,\alpha\beta}^{\tau}\overline{F}_{q,\alpha}^\tau
\overline{F}_{\bar{q},\beta}^\tau\,.
\end{equation}



\begin{thebibliography}{alp}

%\cite{Scimemi:2017etj}
\bibitem{Scimemi:2017etj}
  I.~Scimemi and A.~Vladimirov,
  %``Analysis of vector boson production within TMD factorization,''
  arXiv:1706.01473 [hep-ph].
  %%CITATION = ARXIV:1706.01473;%%
  %2 citations counted in INSPIRE as of 24 Oct 2017

\end{thebibliography}

\end{document}
